\chapter{概率与概率空间}

\section{随机事件与概率}

\subsection{随机事件与概率}

\subsubsection{事件与样本空间}

一般地,我们把实验的每一种可能的结果称为一个\textbf{基本事件}(或称\textbf{样本点}),称所有基本事件的全体为该试验的\textbf{样本空间},记为$\Omega$。

$\Omega$的一个子集合$A$,可以看作是一个随机事件。

从数学的角度看,与试验相关的每个“事件”都可以描述称样本空间$\Omega$的一个子集$A$,反之亦然。

在一次试验中,我们得到了一个结果$\omega \in \Omega$。如果$\omega \in A$,我们就称事件$A$发生了;否则就说$A$没有发生。

如果$\omega \in \Omega$恒成立,我们称$\Omega$为必然事件,其反面为$\Phi$,称为不可能事件。

\subsubsection{古典概型}

古典概型描述了一个随机试验所包含的单位事件都是有限的,且每个单位事件发生的可能性均相等的情况。

\begin{equation}
    P(A) = \frac{|A|}{|\Omega|}
\end{equation}

\subsubsection{事件之间的关系与运算}

事件之间的关系

\begin{itemize}[itemsep=0pt,parsep=0pt]
    \item 事件的包含:$A \subset B$
    \item 事件的相等:$A = B$
    \item 事件的对立:$A \cap A' = \Phi$,$A \cup A' = \Omega$
\end{itemize}

事件之间的运算

\begin{itemize}[itemsep=0pt,parsep=0pt]
    \item 并:$A \cup B$
    \item 交:$A \cap B = AB$
    \item 差:$A - B$
    \item 有限个事件的并:$\bigcup_{i=1}^n A_i$
    \item 有限个事件的交:$\bigcap_{i=1}^n A_i$
\end{itemize}

事件之间的运算法则

\begin{itemize}[itemsep=0pt,parsep=0pt]
    \item 交换律:$A \cup B = B \cup A$,$A \cap B = B \cap A$
    \item 结合律:$(A \cup B) \cup C = A \cup (B \cup C)$,$(A \cap B) \cap C = A \cap (B \cap C)$
    \item 分配律:$A \cup (B \cap C) = (A \cup B) \cap (A \cup C)$,$A \cap (B \cup C) = (A \cap B) \cup (A \cap C)$
    \item 对偶律:$(A \cup B)' = A' \cap B'$,$(A \cap B)' = A' \cup B'$
    \item De Morgan律:$(A \cup B)' = A' \cap B'$,$(A \cap B)' = A' \cup B'$
\end{itemize}